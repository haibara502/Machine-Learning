\documentclass{article}

\usepackage{graphicx}
\usepackage{float}
\usepackage{amsthm}
\usepackage{algorithm}
\usepackage{algorithmicx}
\usepackage{algpseudocode}

\usepackage{caption}

\usepackage{amssymb}
\newenvironment{claim}[1]{\par\noindent\underline{Claim:}\space#1}{}
\newenvironment{guess}[1]{\par\noindent\underline{Guess:}\space#1}{}
\usepackage{amsthm}

\usepackage{geometry}
\geometry{left=2.5cm,right=2.5cm}

\begin{document}
	\section{Warm up: Linear Classifiers and Boolean Functions}
		\begin{enumerate}
			\item It is linearly separable. The linear threshold unit could be:
			\begin{equation}
				y = \left\{ \begin{array}{ll}
					1 & if \quad x_1 + x_3 - x_2 \geq 0 \\
					0 & otherwise \end{array} \right.
			\end{equation}
			\item It is linearly separable. The linear threshold unit could be:
			\begin{equation}
				y = \left\{ \begin{array}{ll}
					1 & if \quad x_1 + x_3 + 2x_2 \geq 2 \\
					0 & otherwise \end{array} \right.
			\end{equation}
			\item It is not linearly separable.
			\item It is not linearly separable.
			\item It is linearly separable. The linear threshold unit could be:
			\begin{equation}
				y = \left\{ \begin{array}{ll}
					1 & if \quad -x_1+x_2-x_3\geq 1 \\
					0 & otherwise \end{array} \right.
			\end{equation}
		\end{enumerate}
	\section{Mistake Bound Model of Learning}
	\begin{enumerate}
		
		\item \begin{enumerate} \item There are only $80$ possible values for $l$. And each $l$ corresponds to a unique function in the concept class. So we can se that 
		\begin{equation}
			|C| = 80
		\end{equation}
		\item Define a function
		\begin{equation}
			g(x_1, x_2) = f_l(x_1,x_2) - y^t
		\end{equation}
		if $g()$ equals zero, no mistake is made here. Otherwise, it makes a mistake.
		\item \begin{enumerate}
		\item If $g(x_1, x_2)$ is greater than zero, showing that function $f$ think it is positive but the true label isn't. In this case, the range of $f$ is bigger than we want. So remove all the functions with length no smaller than $l$.
		\item If $g(x_1, x_2)$ is smaller than zero, showing that function $f$ thinks that it is negative but the true label is positive. In this case, the range of $l$ is smaller than what we want. So remove all the functions whose length is no bigger than $l$ in the concept class.
		\end{enumerate}
		\item \begin{algorithm}[H]
		\caption{Mistake-driven Learning Algorithm}
		\begin{algorithmic}[1]
			\State Initialize $C_0 = C$.
			\State \For{ sample $(x_1^i, x_2^i)$}
				\State Find function $f_k$ whose length $k$ is the middle number among the length of all the functions in current concept class $C_i$.
				\State Check if the function $f_k$ made a mistake.
				\State \If{$f_k$ doesn't make a mistake}
					\State Assign $C_{i + 1} = C_{i}$
					\State Continue to the next sample
				\Else { 
					\State Calculate $g = f_k(x_1^i, x_2^i) - y^i$
					\If {$g > 0$}
						\State Remove all the functions in $C_i$ with length $\geq k$.
					\Else {
						\State Remove all the functions in $C_i$ with length $\leq k$.
					\EndIf}
				} \EndIf
				\If {$|C_i| = 1$}
					\State Break the loop.
				\EndIf
				\State Assign $C_{i + 1} = C_i$
            \EndFor
			
			\State \Return the only remaining function.
        \end{algorithmic}
        \end{algorithm}
		\end{enumerate}
	\item \begin{proof}
		For the Halving algorithm, the algorithm will stop when only $M$ functions in the concept class. Because the remaining  $M$ functions are all perfect experts since they will never be removed from the concept class. Suppose the algorithm made $k$ mistakes before it stops. Then we have:
		\begin{equation}
			M \times 2^k = N
		\end{equation}
		So we know that $k = \log{\frac{N}{M}}$. It means that the mistakes the algorithm will make is no bigger than $\log{\frac{N}{M}}$. That is, the mistake bound of it is $O(\log{\frac{N}{M}})$.
	\end{proof}
	\end{enumerate}
	\section{The perceptron Algorithm and its Variants}
	
	
\end{document}
